% layout
    %% indentation in first paragraph
    \usepackage{indentfirst}
    \setlength{\parindent}{20pt}

    %% allowing landscape page for tables
    \usepackage{pdflscape}
    \newcommand{\blandscape}{\begin{landscape}}
    \newcommand{\elandscape}{\end{landscape}}

    %% font specs
    \addtokomafont{disposition}{\rmfamily}
    
    %% remove hyphenization
    \usepackage[none]{hyphenat}
    
    %% Text justifiying
    \usepackage{ragged2e}


% tables
% - \usepackage{booktabs}
% - \usepackage{longtable}
% - \usepackage{array}
% - \usepackage{multirow}
% - \usepackage{wrapfig}
% - \usepackage{float}
% - \usepackage{colortbl}
% - \usepackage{pdflscape}
% - \usepackage{tabu}
% - \usepackage{threeparttable}
% - \usepackage{threeparttablex}
% - \usepackage[normalem]{ulem}
% - \usepackage{makecell}
% - \usepackage{xcolor}

\usepackage{setspace}  % Para espaçamento
\usepackage{geometry}  % Para controlar o layout da página
\usepackage{lipsum}
\usepackage{fancyhdr}

\geometry{
    a4paper,
    left=3cm,
    right=2cm,
    top=3cm,
    bottom=2cm
}

% Títulos de seções numerados
\setkomafont{section}{\bfseries\Large}
\setkomafont{subsection}{\bfseries\large}
\setkomafont{subsubsection}{\bfseries\normalsize}
\setkomafont{paragraph}{\bfseries\normalsize}
\setkomafont{subparagraph}{\bfseries\normalsize}

% Espaçamento entre títulos e textos
\RedeclareSectionCommand[
  beforeskip=1.5\baselineskip,
  afterskip=0.5\baselineskip]{section}
\RedeclareSectionCommand[
  beforeskip=1\baselineskip,
  afterskip=0.5\baselineskip]{subsection}
\RedeclareSectionCommand[
  beforeskip=0.75\baselineskip,
  afterskip=0.5\baselineskip]{subsubsection}
  
\newcommand{\tituloCentralizado}[1]{
    \begin{center}
        \uppercase{\textbf{#1}}
    \end{center}
}

\newcommand{\inicioCapitulo}{
    \clearpage
    \thispagestyle{empty}
}


% Definições para o conteúdo da folha de rosto
\newcommand{\folhaderosto}{
    \thispagestyle{empty}
    \begin{center}
    \vspace*{-5cm}  
        {\uppercase{\textbf{JOÃO VICTOR BATISTA LOPES}}}

        \vspace{5cm}
        
        {\uppercase{\textbf{EFEITOS DO PERTENCIMENTO DO MUNICÍPIO À FAIXA DE FRONTEIRA BRASILEIRA SOBRE A VIOLÊNCIA}}}
        
        
        \vspace{3cm}
        \end{center}

        \hspace{8cm}
        \parbox{7cm}{
        \setstretch{1.0} % Define o espaçamento simples
        Dissertação apresentada ao Programa de Pós-Graduação em Economia da Faculdade de Economia, Administração, Atuária e Contabilidade da Universidade Federal do Ceará, como requisito parcial para obtenção do título de Mestre em Economia.
        
        \vspace{0.5cm}
        
        Orientador: Prof. Dr. José Raimundo de Araújo Carvalho Junior
        }
        
        \vfill
        \begin{center}
        
        Fortaleza – CE \\
        2024
    \end{center}
}

% Definições para o conteúdo da folha de aprovação
\newcommand{\folhadeaprovacao}{
    \thispagestyle{empty}
    \begin{center}
    \vspace*{-5cm}
        \uppercase{\textbf{JOÃO VICTOR BATISTA LOPES}}
        
        \vspace{5cm}
        
        \uppercase{\textbf{EFEITOS DO PERTENCIMENTO DO MUNICÍPIO À FAIXA DE FRONTEIRA BRASILEIRA SOBRE A VIOLÊNCIA}}

        \vspace{3cm}
    \end{center}
    
    \hspace{8cm}
    \parbox{7cm}{
        \setstretch{1.0} % Define o espaçamento simples
        Dissertação apresentada ao Programa de Pós-Graduação em Economia da Faculdade de Economia, Administração, Atuária e Contabilidade da Universidade Federal do Ceará, como requisito parcial para obtenção do título de Mestre em Economia.
    }
    
    \vspace{0.5cm}
    
    \noindent Aprovada em \today.
    
    \begin{center}
        \textbf{Banca Examinadora} \\
        
        \vspace{1cm}
        
        \underline{\hspace{10cm}} \\
        Prof. Dr. Nome do Presidente da Banca \\
        Instituição
        
        \underline{\hspace{10cm}} \\
        Prof. Dr. Nome do Membro 1 \\
        Instituição
        
        \underline{\hspace{10cm}} \\
        Prof. Dr. Nome do Membro 2 \\
        Instituição
    \end{center}
}

% Definições para o conteúdo da dedicatória
\newcommand{\dedicatoria}{
    \thispagestyle{empty}
    \vspace*{10cm} % Ajusta a posição para começar abaixo do meio da folha
    
    \hspace{8cm} % Recuo de 8 cm da margem esquerda
    \parbox{7cm}{
        \setstretch{1.5} % Espaçamento de 1,5 entre linhas
        % Texto da dedicatória
        A todos que me apoiaram nesta jornada, com carinho e dedicação.
    }
}

% Definições para o conteúdo dos agradecimentos
\newcommand{\agradecimentos}{
    \thispagestyle{empty}
    
    % Título "AGRADECIMENTOS" centralizado
    
    \vspace*{-5cm}
    \begin{center}
        \uppercase{\textbf{AGRADECIMENTOS}}
    \end{center}
    
    \vspace{1cm} % Espaçamento entre o título e o início do texto
    
    % Texto dos agradecimentos com fonte tamanho 12 e espaçamento de 1,5 entre linhas
    \setstretch{1.5} % Define o espaçamento de 1,5
    \noindent % Início do texto justificado
    O presente trabalho foi realizado com apoio da Coordenação de Aperfeiçoamento de Pessoal de Nível Superior – Brasil (CAPES) – Código de Financiamento 001.
    
    \vspace{0.5cm}
    
    Gostaria de agradecer a todos que contribuíram para a realização deste trabalho, em especial...
}

% Definições para epígrafe curta (até 3 linhas)
\newcommand{\epigrafeCurta}{
    \thispagestyle{empty}
    \vspace*{10cm} % Ajuste para começar abaixo do meio da página
    
    \hspace{8cm} % Recuo de 8 cm da margem esquerda
    \parbox{7cm}{
        \setstretch{1.5} % Espaçamento de 1,5 entre linhas
        “Texto da epígrafe, entre aspas” % Texto da epígrafe em itálico e entre aspas
    }
    
    \vspace{0.5cm}
    \hfill — Autor, Ano, p. XX
}

% Definições para epígrafe longa (mais de 3 linhas)
\newcommand{\epigrafeLonga}{
    \thispagestyle{empty}
    \vspace*{10cm} % Ajuste para começar abaixo do meio da página
    
    \hspace{4cm} % Recuo de 4 cm da margem esquerda
    \parbox{11cm}{
        \setstretch{1.0} % Espaçamento simples entre linhas
        \fontsize{10}
        Texto da epígrafe, sem aspas % Texto da epígrafe (sem aspas)
    }
    
    \vspace{0.5cm}
    \hfill — Autor, Ano, p. XX
}

% Definições para o conteúdo do resumo
\newcommand{\resumo}{ 
    \thispagestyle{empty}
    
    % Título "RESUMO" centralizado
    \vspace*{-5cm}
    \begin{center}
        \uppercase{\textbf{RESUMO}}
    \end{center}
    
    \vspace{0.5cm} % Espaçamento entre o título e o texto do resumo
    
    % Texto do resumo justificado, sem recuo e com espaçamento de 1,5
    \noindent % Remove recuo do primeiro parágrafo
    \justifying
    Texto do resumo objetivos, metodologia, resultados e conclusões
    
    \vspace{0.5cm} % Espaço entre o resumo e as palavras-chave
    
    % Palavras-chave
    \noindent \textbf{Palavras-chave:}  Inclua suas palavras-chave separadas por ponto e vírgula, terminando com ponto
}

